\documentclass[a4paper,11pt]{article}
\usepackage[french]{babel}
\usepackage{xspace}
\usepackage{luamplib}

\usepackage[a4paper, includefoot, textwidth=6.5in, textheight=24.956cm,
            hmarginratio=1:1, vmarginratio=1:1,
            footnotesep=1.5\baselineskip]{geometry}

\usepackage{mathtools}
\usepackage[warnings-off={mathtools-colon},
            warnings-off={mathtools-overbracket},
            math-style=french]{unicode-math}
\usepackage{fontspec}
\usepackage{microtype}

\makeatletter
\def\@maketitle{%
  \begin{center}%
    \let\footnote\thanks
    {\bfseries \@title \par}%
    \vskip .5em
    \@author
  \end{center}%
  \par \vskip .5em}

\renewenvironment{abstract}%
  {\quotation\noindent\ignorespaces}%
  {\endquotation}

% Multiple footnotes
\let\oldFootnote\footnote
\newcommand\nextToken\relax
\renewcommand\footnote[1]{%
    \oldFootnote{#1}\futurelet\nextToken\isFootnote}
\newcommand\isFootnote{%
    \ifx\footnote\nextToken\textsuperscript{,}\fi}


\newcommand{\mytilde}{\raise.17ex\hbox{$\scriptstyle\mathtt{\sim}$}\xspace}

\AtBeginDocument{\let\ge\geqslant \let\le\leqslant}
\makeatother

\usepackage[unicode,hyperfootnotes=false,hidelinks]{hyperref}
\hypersetup{%
  pdftitle={B.a-ba des a.c.u.},
  pdfauthor={Lionel Vidal}
}
\usepackage{bookmark}
\title{B.a-ba des a.c.u.}
\author{Lionel \bsc{V\kern-1pt idal}}

\begin{document}
\maketitle
\begin{abstract}
  Un court aide-mémoire%
  \footnote{Pour des définitions précises et des exemples
    ou contre-exemples pertinents qui justifient l'existence des
    différentes notions, on pourra consulter le
    {\slshape Cours d'algèbre}, de D.~Perrin (le célèbre livre bleu
    des agrégatifs).}%
  \footnote{Et le remarquable
    {\slshape Théorie algébrique des nombres}, de P.~Samuel.}%
  \footnote{Et le touffu polycopié
    {\slshape Préparation à l'écrit 1989}, de J.-M.~Exbrayat.}
  sur une partie de la famille des anneaux
  commutatifs unitaires (a.c.u.) et les relations particulières
  qui unissent ses membres.
\end{abstract}

\bigskip
\noindent Les flèches se lisent ici comme des
implications, et en pointillées, elles soulignent une relation de
définition : par exemple, si un anneau est factoriel, alors il est
intégralement clos, ou encore, un anneau de Dedekind est par définition
nœthérien. Les anneaux sur fond grisé sont intègres par définition. Les
notes précisent les domaines de validité de certaines propriétés
arithmétiques utiles.

\bigskip
\begin{center}
\mbox{\begin{mplibcode}
  input boxes ;
  beginfig(0) ;

  vgap := 50 ; hgap := 60 ;
  defaultdx := 8 ; defaultdy := 8 ;

  boxit.ae(btex euclidien etex) ;
  boxit.ap(btex principal etex) ;
  boxit.ac(btex intégralement clos etex) ;
  boxit.af(btex factoriel etex) ;
  boxit.ad(btex de Dedekind etex) ;
  boxit.ab(btex de Bézout\up{$*$ (6)} etex);
  boxit.an(btex nœthérien\up{$*$} etex);
  boxit.ag(btex à PGCD etex);
  boxit.ai(btex intègre\up{(1, 2)} etex);

  ae.s - ap.n = ap.s - af.n = af.s - ac.n = ac.s - ai.n = (0, vgap) ;
  af.w - ad.e = ab.w - af.e = (hgap, 0);
  an.e - ai.w = (whatever, 0); an.c - ad.c = (0, whatever);
  ag.w - ai.e = (whatever, 0); ag.c - ab.c = (0, whatever);

  fixsize(ae, ap, ad, af, ab, ac, ai);
  forsuffixes s=ae,ap, ad, af, ab, ac, ai:
    fill bpath.s withcolor 0.9white ;
  endfor ;
  drawboxed(ae, ap, af, ac, ad, ab, an, ag, ai);

  drawarrow ae.s -- ap.n ;
  drawarrow ap.s -- af.n ;
  drawarrow ap.s -- ad.ne ;
  drawarrow ap.s -- ab.nw ;
  drawarrow af.s -- ac.n ;
  drawarrow af.se -- ag.n ;
  drawarrow ad.s -- an.n dashed evenly ;
  drawarrow ad.s -- ac.nw dashed evenly ;
  drawarrow ab.s -- ag.n ;
  drawarrow ab.s -- ac.ne ;

  boxit.adaf();
  boxit.afab();
  boxit.anaiag();
  boxit.aiag();
  adaf.nw - ad.nw = (-defaultdx, defaultdy);
  adaf.se - af.se = (defaultdx, -defaultdy);
  afab.nw - af.nw = (-defaultdx, defaultdy+4);
  afab.se - ab.se = (defaultdx, -defaultdy-4);
  anaiag.nw - an.nw = (-defaultdx, defaultdy);
  anaiag.se - ag.se = (defaultdx, -defaultdy);
  aiag.nw - ai.nw = (-defaultdx, defaultdy+4);
  aiag.se - ag.se = (defaultdx+4, -defaultdy-4);
  fixsize(adaf, afab, anaiag, aiag);
  drawboxed(adaf, afab, anaiag, aiag);

  drawarrow ad.n -- ap.sw cutbefore bpath.adaf ;
  drawarrow ab.n -- ap.se cutbefore bpath.afab ;
  drawarrow ai.n -- ac.s cutbefore bpath.aiag ;
  drawarrow an.n -- af.sw cutbefore bpath.anaiag ;

  x.star = xpart ab.c - 10 ; y.star = ypart ap.c ;
  drawarrow z.star -- ap.e ;
  label.rt(btex $*$ etex, z.star);

  label.rt(btex \up{(3, 4, 5)} etex, aiag.ne);

  endfig;
\end{mplibcode}}
\end{center}
\bigskip

\renewcommand{\labelenumi}{(\arabic{enumi}) }
\begin{enumerate}
\item la relation de divisibilité sur A/\mytilde (où \mytilde est la
  relation d'équivalence qui lie deux éléments associés) est un ordre
  (ce n'est qu'un pré-ordre si l'anneau $A$ n'est pas intègre).

\item tout élément premier est irréductible.

\item lemme de Gauss : si $a$ divise $bc$ et si $a$ est premier
  avec $b$, alors $a$ divise $c$.

\item lemme d'Euclide : si $p$ irréductible divise $ab$,
  alors $p$ divise $a$ ou $b$.

\item un élément est premier si et seulement si il est irréductible.

\item deux éléments sont étrangers si et seulement si ils sont premiers
  entre eux.
\end{enumerate}

\end{document}
