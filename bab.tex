\documentclass[a4paper,11pt]{article}
\usepackage[french]{babel}
\usepackage[np]{numprint}
\usepackage{xspace}

\usepackage[a4paper, includefoot, textwidth=6.5in, textheight=24.956cm,
            hmarginratio=1:1, vmarginratio=1:1,
            footnotesep=1.5\baselineskip]{geometry}
\usepackage[explicit]{titlesec}
\titleformat{\section}[runin]
  {\normalfont\bfseries}
  {\thesection.}{\wordsep}{#1.}
\titlespacing{\section}{0pt}{*3}{*1.5}

\usepackage{mathtools}
\usepackage[warnings-off={mathtools-colon},
            warnings-off={mathtools-overbracket},
            math-style=french]{unicode-math}
\usepackage{fontspec}
\usepackage{microtype}

\makeatletter
\def\@maketitle{%
  \begin{center}%
    \let\footnote\thanks
    {\bfseries \@title \par}%
    \vskip .5em
    \@author
  \end{center}%
  \par \vskip .5em}

\newcommand{\N}{\ensuremath{\symbb{N}}\xspace}
\newcommand{\Z}{\ensuremath{\symbb{Z}}\xspace}
\newcommand{\Q}{\ensuremath{\symbb{Q}}\xspace}
\AtBeginDocument{\let\ge\geqslant \let\le\leqslant}
\makeatother

\usepackage[unicode,hyperfootnotes=false,hidelinks]{hyperref}
\hypersetup{%
  pdftitle={Rudis indigestaque moles},
  pdfauthor={Lionel Vidal}
}
\usepackage{bookmark}
\title{Rudis indigestaque moles, ou (ru)bric-à-brac mathématique}
\author{Lionel \bsc{V\kern-1pt idal}}

\begin{document}
\maketitle

\section{Les trois deux}

Comment représenter un nombre entier strictement positif quelconque avec
seulement trois $2$ et des symboles mathématiques ?

On écrit cet entier $N$ sous la forme :
\[ N = -\log_2 \log_2 \sqrt{\sqrt{\ldots\sqrt{2}}} \]
où le nombre de radicaux est égal au nombre entier voulu ! Une démonstration
de cette formule serait à demander à un lycéen pour vérifier sa maîtrise des
règles de calcul sur les logarithmes\dots

\section{Question très bête}

Quel est le triangle qui a la plus grande aire : celui de côtés $5$, $5$,
$6$ ou celui de côtés $5$, $5$, $8$ ?

\section{C'est sûrement vrai, j'ai fait de (très) nombreux tests}

Pour tout $n\in\N$, les nombres $n^{17}+9$ et $(n+1)^{17}+9$ sont premiers
entre eux... ou pas !
Le premier contre-exemple\footnote{
  Voir le livre, {\slshape Merveilleux nombres premiers}, de J.-P.~Delahaye.}
n'apparait que pour :
\[ n = 8424432925592889329288197322308900672459420460792433\,. \]

\section{Citation}

Attribuée à Lipman Bers, sur mon objet mathématique préféré :
\begin{quotation}\itshape\noindent
  God, if She exists, created the natural numbers $\{0,1,2,\ldots\}$,
  and compact Riemann surfaces. The rest of mathematics is man made.
\end{quotation}

\section{Citation}

Jugement définitif de Robert A. Heinlein, qui montre que les auteurs de
science-fiction ont une vision souvent très claire des mathématiques :
\begin{quotation}\itshape\noindent
  Anyone who cannot cope with mathematics is not fully human. At best, he
  is a tolerable subhuman who has learned to wear his shoes, bathe, and
  not make messes in the house.
\end{quotation}

\section{Un problème renversant}

On fixe un point $I$ sur une sphère $S(O,r)$ de l'espace affine de dimension
$3$, et on considère l'ensemble des tétraèdres $IABC$ inscrits dans la sphère
et tri-rectangles en $I$. On demande\footnote{
  Problème lu dans l'excellent polycopié,
  {\slshape Compléments de géométrie}, de J.-M.~Exbrayat.}
le lieu du centre de gravité $G$ des triangles variables $ABC$\dots

\smallskip
En analytique, c'est cauchemardesque, mais c'est très simple si on pense
à renverser le point de vue pour regarder la sphère depuis le fauteuil
variable du repère orthonormé centré en $I$, qui suit les arêtes du
tétraèdre.

Dans ce repère, on a $A(\alpha,0,0)$, $B(0,\beta,0)$, $C(0,0,\gamma)$,
où $\alpha$, $\beta$ et $\gamma$ sont trois réels.
On a donc $G(\alpha/3, \beta/3, \gamma/3)$, et comme $O$ est à
l'intersection des trois plans médiateurs des segments $[IA]$, $[IB]$ et
$[IC]$, on a aussi $O(\alpha/2, \beta/2, \gamma/2)$.
Mais $I$ et $O$ sont fixes, donc $G$ est fixe !

\section{La suite des anciens}

Moins de cinquante ans, passez votre chemin\dots{}
Il s'agit de déterminer les nombres suivants de la suite :
\[ 6 ;\quad 2;\quad 5;\quad 5; \quad4;\quad 5;\quad 6; \quad 3;\quad ? \]
Impossible à trouver, mais sentant bon la nostalgie technologique :
pensez au nombre de segments utilisés par un affichage électronique,
en commençant par $0$\dots

\section{6accdæ13eff7i3l9n4o4qrr4s8t12ux}

Ou comment cacher pour un temps une découverte importante tout en
assurant la paternité. Newton (1642-1727) dissimule son travail sur les
équations différentielles dans cette anagramme célèbre qui se lit
(avec une erreur sur le nombre de \emph{t} !) :
\begin{quotation}\itshape\noindent
  data æquatione quotcunque fluentes quantitates involvente fluxiones
  invenire et vice versa.
\end{quotation}

\noindent
Une traduction libre en termes modernes pourrait en être :
{\itshape les lois naturelles s'expriment par des équations
différentielles.}\footnote{%
  Voir l'introduction du remarquable livre,
  {\slshape Équations différentielles ordinaires}, de V.~Arnold.}

\end{document}
