\documentclass[a4paper,11pt]{article}
\usepackage[french]{babel}
\usepackage{xspace}

\usepackage[a4paper, includefoot, textwidth=6.5in, textheight=24.956cm,
            hmarginratio=1:1, vmarginratio=1:1,
            footnotesep=1.5\baselineskip]{geometry}

\usepackage{mathtools}
\usepackage[warnings-off={mathtools-colon},
            warnings-off={mathtools-overbracket},
            math-style=french]{unicode-math}
\usepackage{fontspec}
\usepackage{microtype}

\makeatletter
\def\@maketitle{%
  \begin{center}%
    \let\footnote\thanks
    {\bfseries \@title \par}%
    \vskip .5em
    \@author
  \end{center}%
  \par \vskip .5em}

\renewenvironment{abstract}%
  {\quotation\noindent\ignorespaces}%
  {\endquotation}

\newcommand{\N}{\ensuremath{\mathbb{N}}\xspace}
\newcommand{\R}{\ensuremath{\mathbb{R}}\xspace}
\AtBeginDocument{\let\ge\geqslant \let\le\leqslant}
\newcommand{\proclaim}[1]{\medskip\noindent{\bfseries #1.}\enspace}
\newcommand{\proof}{\medskip\noindent{\slshape Démonstration}.\enspace}
\makeatother

\usepackage[unicode,hyperfootnotes=false,hidelinks]{hyperref}
\hypersetup{%
  pdftitle={Théorème des valeurs intermédiaires et base 13},
  pdfauthor={Lionel Vidal}
}
\usepackage{bookmark}

\title{Théorème des valeurs intermédiaires et base 13}
\author{Lionel \bsc{V\kern-1pt idal}}
\begin{document}
\maketitle
\begin{abstract}
  Où une remarquable fonction, définie par un détour en base $13$, sert de
  contre-exemple fort à la réciproque du théorème des valeurs intermédiaires.
\end{abstract}

\bigskip
Voici comment Frédéric Testard présente le théorème des valeurs intermédiaires,
dans son remarquable livre, {\slshape Analyse mathématique, la maîtrise de
l'implicite} :
\begin{quotation}\itshape
\noindent
Élémentaire par son caractère intuitif, [\dots], mais profond dans sa preuve
qui, quoique simple, s'appuie sur la propriété topologique la plus importante
de $\R$ : la complétude, et profond aussi dans ces conséquences, car c'est lui
qui garantit l'existence d'une multitude d'objets mathématiques tels que les
fonctions racines, exponentielles, etc.
\end{quotation}

\medskip
Mais rappelons-en d'abord l'énoncé ainsi qu'une démonstration possible :

\proclaim{Théorème des valeurs intermédiaires}
Soit $f : [a,b]\to \R$ une fonction réelle continue.  Si $y$ est compris
entre $f(a)$ et $f(b)$, il existe $c\in[a,b]$ tel que $f(c)=y$.

\proof
L'intervalle $[a,b]$ est connexe, donc son image par $f$ continue est connexe.
Comme les seuls connexes de $\R$ sont les intervalles et que $f(a)$ et $f(b)$
sont dans $f([a,b])$, tout nombre $y$ compris entre $f(a)$ et $f(b)$ est aussi
élément de $f([a,b])$.

\smallskip\noindent
Cette démonstration est aussi courte qu'efficace, mais il faut prendre
garde alors, comme le fait remarquer F.~Testard dans son ouvrage cité, à ne
pas utiliser ce théorème pour démontrer que les connexes de $\R$ sont les
intervalles, sous peine de raisonnement circulaire !

\medskip
Considérons maintenant la réciproque de ce théorème : une fonction $f$ vérifiant
la propriété des valeurs intermédiaires est-elle continue ? La réponse est
négative, comme le montre classiquement la fonction définie sur $\R$ par :
\[
  f(x) \coloneq
  \begin{cases}
    \sin\left(\frac{1}{x}\right), & \text{si $x\neq0$ ;} \\
    0,                            & \text{si $x=0$.}
  \end{cases}
\]

Ce contre-exemple, suffisant bien sûr, semble toutefois un peu faible dans la
mesure où $f$ n'est discontinue qu'en un seul point. Pour trouver mieux, on
peut chercher quelle condition naturelle devrait vérifier $f$, en plus de la
propriété des valeurs intermédiaires, pour assurer sa continuité.

\noindent Par exemple :

\proclaim{Théorème}
Si $f:\R\to\R$ est une fonction injective vérifiant la propriété des
valeurs intermédiaires, alors $f$ est continue sur $\R$.

\proof
Supposons que $f$ soit discontinue en un point $x$. Il existe alors une
suite $(x_n)$ convergeant vers $x$, telle que la suite des $f(x_n)$ ne
converge pas vers $f(x)$. Donc, il existe $\epsilon>0$ et une sous-suite
$(x_{n_k})$ de $(x_n)$, telle que pour tout $n_k$,
$f(x_{n_k})\not\in\; ]f(x)-\epsilon, f(x)+\epsilon\,[$.

Quitte à extraire une sous-suite de $(x_{n_k})$, notée encore $(x_{n_k})$,
on peut considérer que les $f(x_{n_k})$ sont,
soit tous inférieurs à $f(x)-\epsilon$,
soit tous supérieurs à $f(x)+\epsilon$.
Le raisonnement étant le même dans les deux cas, supposons par exemple
que pour tout $n_k$, $f(x_{n_k})\le f(x)-\epsilon<f(x)$.

Pour tout $n_k$, il existe alors, d'après la propriété des valeurs
intermédiaires, un réel $c_k$ compris entre $x_{n_k}$ et $x$, tel que
$f(c_k)=f(x)-\epsilon$. Mais comme $f$ est injective, tous les $c_k$ sont
égaux à un réel~$c$. Et comme la suite $(x_{n_k})$ tend vers $x$, on en
déduit par encadrement que $x=c$.
Ce qui contredit : $f(c)=f(x)-\epsilon\neq f(x)$.

\medskip
Peut-être faudrait-il considérer une fonction qui ne soit vraiment pas
injective ? Que penser alors de la propriété suivante, où $f$ est une
fonction définie sur $\R$ :

\proclaim{Propriété (P)}
Pour tout intervalle ouvert non vide $I\subseteq\R$, $f(I)=\R$.

Il n'est pas évident qu'une fonction vérifiant (P) existe, mais supposons
que cela soit le cas. Alors pour sûr, $f$ n'est pas injective ! Et elle
vérifie évidemment la propriété des valeurs intermédiaires.
Que dire de la continuité d'une telle fonction ?

\proclaim{Théorème}
Si $f : \R\to\R$ est une fonction vérifiant (P),
alors $f$ est discontinue en tout point de~$\R$.

\proof
Supposons que $f$ soit continue en $x\in\R$, et considérons un intervalle
$V$ ouvert de longueur finie contenant $f(x)$. Il existe alors un voisinage
ouvert $U$ de $x$ tel que $f(U)\subseteq V$. Mais comme $f$ vérifie (P),
$f(U)=\R$, et donc $\R\subseteq V$, ce qui contredit la longueur finie de~$V$.

\medskip
Une fonction vérifiant (P) constituerait donc, en un sens, un contre-exemple
fort à la réciproque du théorème des valeurs intermédiaires. Mais comment
trouver un tel monstre ?

En utilisant la représentation des nombres réels en base $13$
(quelle idée !), le mathématicien John~H.~Conway
a construit dans ce but l'incroyable fonction présentée ci-après.

\medskip
Tout nombre réel peut s'écrire de façon unique en base $13$ sous la forme
\[
  \pm\, c_0c_1\ldots c_k\,.\,c_{k+1}c_{k+2}\ldots
\]
où $(c_n)_{n\in\N}$ est une suite d'éléments de
$\{0, 1, \ldots, 9, A, B, C\}$, telle que si $c_0=0$, alors $k=0$ (la
partie entière ne comporte pas de zéros inutiles), et telle que pour tout
entier $n\ge0$, il existe un entier $m>n$ tel que $c_m\neq C$ (l'expansion
trédécimale ne peut pas se terminer par une suite infinie de $C$).
Ce type d'écriture unique, normalisée, existe bien sûr pour toute base.

\smallskip
On définit alors une fonction $f$ réelle de la façon suivante :
\begin{itemize}
\item pour tout $x$ réel écrit comme ci-dessus en base 13,
  le signe et le point trédécimal sont ignorés ;
\item si la fin de l'expansion trédécimale de $x$ est de la forme
  $Ax_1x_2\ldots x_nCy_1y_2\ldots$, où tous les $x_i$ et tous les $y_i$
  sont des chiffres décimaux, alors $f(x)=x_1x_2\ldots x_n.y_1y_2\ldots$,
  considéré comme un nombre en base $10$ ;
\item de façon similaire, si la fin de l'expansion de
  $x$ est de la forme $Bx_1x_2\ldots x_nCy_1y_2\ldots$, alors
  $f(x)=-x_1x_2\ldots x_n.y_1y_2\ldots$ ;
\item sinon, $f(x)=0$.
\end{itemize}

\smallskip
Quelques exemples pour fixer les idées :
\begin{align*}
  f(-1.23A3C14159\ldots_{13}) &= 3.14159\ldots \\
  f(+123B4C567.00\ldots_{13}) &= -4.567 \\
  f(+1C23A4.500\ldots_{13})   &= 0
\end{align*}

Soient maintenant un intervalle ouvert $I$ non vide de $\R$, $c$ un
élément de $I$ et $y$ un réel quelconque.
Partant de l'écriture normalisée de $y$ en base $10$ et de $c$ en base $13$,
on construit alors un réel $c'$ en base $13$, en changeant, à partir
d'un certain rang, la fin de l'expansion suivant le point
trédécimale de $c$ en $Ay$ ou $By$, suivant le signe de $y$, et en remplaçant
le point décimal de $y$ par $C$.

En effectuant cette modification suffisamment loin dans l'expansion
de $c$, on peut s'assurer que $c'$ est également dans
l'intervalle $I$. Et comme par construction $f(c')=y$, on en déduit que
$f$ vérifie la propriété (P).

\begin{center}\slshape
La fonction $f$ vérifie donc la propriété des valeurs intermédiaires en
étant partout discontinue !
\end{center}

\end{document}
