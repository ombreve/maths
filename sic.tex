\documentclass[a4paper,11pt]{article}
\usepackage[french]{babel}
\usepackage{xspace}

\usepackage[a4paper, includefoot, textwidth=6.5in, textheight=24.956cm,
            hmarginratio=1:1, vmarginratio=1:1,
            footnotesep=1.5\baselineskip]{geometry}
\usepackage[explicit]{titlesec}
\titleformat{\section}[runin]
  {\normalfont\bfseries}
  {\thesection.}{\wordsep}{#1.}
\titlespacing{\section}{0pt}{*3}{*1.5}

\usepackage{mathtools}
\usepackage[warnings-off={mathtools-colon},
            warnings-off={mathtools-overbracket},
            math-style=french]{unicode-math}
\usepackage{fontspec}
\usepackage{microtype}

\makeatletter
\def\@maketitle{%
  \begin{center}%
    \let\footnote\thanks
    {\bfseries \@title \par}%
    \vskip .5em
    \@author
  \end{center}%
  \par \vskip .5em}

\renewenvironment{abstract}%
  {\quotation\noindent\ignorespaces}%
  {\endquotation}

\newcommand{\R}{\ensuremath{\symbb{R}}\xspace}
\newcommand{\g}{\ensuremath{\gamma}\xspace}
\newcommand{\bg}{\ensuremath{b_{\gamma}}\xspace}
\newcommand{\eg}{\ensuremath{\epsilon_{\gamma}}\xspace}
\DeclareMathOperator{\sinc}{sinc}
\DeclareMathOperator{\sign}{sign}
\AtBeginDocument{\let\ge\geqslant \let\le\leqslant}
\makeatother

\usepackage[unicode,hyperfootnotes=false,hidelinks]{hyperref}
\hypersetup{%
  pdftitle={Les surprises du sinus cardinal},
  pdfauthor={Lionel Vidal}
}
\usepackage{bookmark}
\title{Les surprises du sinus cardinal}
\author{Lionel \bsc{V\kern-1pt idal}}

\begin{document}
\maketitle
\begin{abstract}
  Où une remarquable suite d'intégrales de sinus cardinaux\footnote{%
    Voir l'article : {\slshape Some Remarkable Properties of Sinc and
    Related Integrals}, de D.~Borwein et J.~M.~Borwein, dont on suit ici
    une partie des développements.}
  peut inciter, au mépris de toute prudence mathématique, à généraliser
  hâtivement.
\end{abstract}

\section{Une suite peu commune}

On définit sur \R la fonction continue, dite sinus cardinal, par :
\[
  \sinc(x) \coloneq
  \begin{cases}
    \dfrac{\sin x}{x}, & \text{si $x\neq 0$ ;} \\
    1,                & \text{si $x=0$.}
  \end{cases}
\]
Un logiciel de calcul formel donne alors :
\begin{gather*}
  \int_0^\infty \sinc(x)\,dx
    = \frac{\pi}{2} \,, \\
  \int_0^\infty \sinc(x)\sinc\left(\frac{x}{3}\right)\,dx
    = \frac{\pi}{2} \,, \\
  \int_0^\infty \sinc(x)\sinc\left(\frac{x}{3}\right)
    \sinc\left(\frac{x}{5}\right)\,dx
    = \frac{\pi}{2} \,, \\
  \vdots \\
  \int_0^\infty \sinc(x)\sinc\left(\frac{x}{3}\right)
    \sinc\left(\frac{x}{5}\right)\ldots
    \sinc\left(\frac{x}{13}\right)\,dx
    = \frac{\pi}{2} \,,
\end{gather*}
La valeur constante est déjà très remarquable, mais le plus
extraordinaire est à venir :
\[
  \int_0^\infty \sinc(x)\sinc\left(\frac{x}{3}\right)
    \sinc\left(\frac{x}{5}\right)\ldots
    \sinc\left(\frac{x}{15}\right)\,dx
  = \frac{467807924713440738696537864469}{935615849440640907310521750000}
    \pi \,.
\]
Il semble que le logiciel ait un sérieux {\it bug}\dots{}
Sinon, comment expliquer une valeur aussi inattendue ?

\section{Préliminaires}

Soient $a_0$, $a_1$, $\ldots$, $a_n$ des nombres réels, avec $n\ge 1$.
Pour chacun des $2^n$ $n$-uplets
$\g=(\g_1, \ldots, \g_n) \in \{-1;1\}^n$, on définit :
\[
  \bg \coloneq a_0 + \sum_{k=1}^{n} \g_k\,a_k\,,\quad\text{et}\quad
  \eg \coloneq \prod_{k=1}^{n} \g_k \,.
\]

Par simple développement de produit, on remarque alors que :
\[
  e^{a_0 x}\prod_{k=1}^n (e^{a_k x}-e^{-a_k x})
  = \sum_\g \eg\,e^{\bg x} \,.
\]

Mais d'une part, quand $x$ tend vers $0$, on a :
\[
  e^{a_0 x} = 1 + O(x) \,,\quad\text{et}\quad
  e^{a_k x}-e^{-a_k x} = 2a_k x + O(x^2) \,;
\]
et donc :
\[
  e^{a_0 x}\prod_{k=1}^n (e^{a_k x}-e^{-a_k x})
  = 2^n \left(\prod_{k=1}^{n}a_k\right)x^n + O(x^{n+1}) \,.
\]

D'autre part, toujours quand $x$ tend vers $0$, on a :
\[
  \sum_\g \eg\,e^{\bg x}=\left(\sum_\g \eg\right) +
  \left(\sum_\g \eg\,\bg\right)\frac{x}{1!} + \ldots +
  \left(\sum_\g \eg\,\bg^n\right)\frac{x^n}{n!} + O(x^{n+1}) \,.
\]

D'où, par identification des coefficients :
\begin{equation}\label{eq:sumegbg}
  \sum_\g \eg\,\bg^r =
  \begin{cases}
    0\,, & \text{pour $r=0$, $\ldots$, $n-1$ ;} \\
    2^n\,n! \displaystyle\prod_{k=1}^{n}a_k\,, & \text{pour $r=n$.}
  \end{cases}
\end{equation}

En utilisant la même remarque que précédemment, on a aussi :
\begin{align*}
  \prod_{k=0}^n \sin(a_k x) &= \frac{1}{(2i)^{n+1}}
    (e^{ia_0 x}-e^{-ia_0 x})\prod_{k=1}^n (e^{ia_k x}-e^{-ia_k x}) \\
  &= \frac{1}{(2i)^{n+1}} \sum_\g \eg(e^{i\bg x}-(-1)^n e^{-i\bg x}) \\
  &= \frac{1}{2^{n+1}} e^{-i(n+1)\frac{\pi}{2}}
     \sum_\g \eg(e^{i\bg x}+e^{-i(\bg x-(n+1)\pi)}) \\
  &= \frac{1}{2^{n+1}} \sum_\g \eg(e^{i(\bg x-(n+1)\frac{\pi}{2})} +
     e^{-i(\bg x-(n+1){\pi\over2})}) \,,
\end{align*}
et donc finalement :
\begin{equation}\label{eq:prodsin}
  \prod_{k=0}^n \sin(a_k x) =
  \frac{1}{2^n}\sum_\g \eg\cos\left(\bg x-(n+1)\frac{\pi}{2}\right)\,.
\end{equation}

\section{Intégration}

Maintenant que l'on sait, grâce à (\ref{eq:prodsin}), transformer un produit
de sinus en somme, on peut s'essayer à son intégration !
On écrit :
\[
  \int_0^\infty\prod_{k=0}^n \frac{\sin(a_k x)}{x}\,dx =
  \frac{1}{2^n}\int_0^\infty \frac{1}{x^{n+1}}C_n(x)\,dx\,,
  \quad\text{où}\quad
  C_n(x) \coloneq \sum_\g \eg\cos\left(\bg x-(n+1)\frac{\pi}{2}\right)\,.
\]

La fonction $C_n$ est entière, bornée sur $\R$, et admet un zéro d'ordre
$n+1$ en $x=0$. On peut donc intégrer $n$ fois par parties le membre
de droite de la dernière égalité en garantissant la convergence des
intégrales successives :
\begin{align*}
  \int_0^\infty \frac{1}{x^{n+1}}C_n(x)\,dx
  &= -\int_0^\infty - \frac{1}{nx^n}
     \left(\sum_\g-\eg\,\bg\sin\left(\bg x-(n+1)\frac{\pi}{2}\right)\right)
     \,dx \\
  &= \frac{1}{n}\int_0^\infty \frac{1}{x^n}
     \left(\sum_\g \eg\,\bg\cos\left(\bg x-n\frac{\pi}{2}\right)\right)
     \,dx \\
  &= \frac{1}{n}\int_0^\infty \frac{1}{x^n} C_{n-1}(x)\,dx \\
  &\vdots \\
  &= \frac{1}{n!}\int_0^\infty \frac{1}{x} C_0(x)\,dx \\
  &= \frac{1}{n!}\int_0^\infty \frac{1}{x}
     \left(\sum_\g \eg\,\bg^n\cos\left(\bg x-\frac{\pi}{2}\right)\right)
     \,dx \\
  &= \frac{1}{n!}\sum_\g \eg\,\bg^n \int_0^\infty \frac{\sin(\bg x)}{x}
     \,dx \,.
\end{align*}

On suppose connu le résultat classique :
\[
  \int_0^\infty \frac{\sin(x)}{x}\,dx = \frac{\pi}{2}\,,
  \quad\text{d'où}\quad
  \int_0^\infty \frac{\sin(\bg x)}{x}\,dx = \frac{\pi}{2}\sign(\bg) \,,
\]
où la fonction $\sign$ est définie par :
\[
  \sign(x) \coloneq
  \begin{cases}
    1  & \text{si $x>0$,} \\
    0  & \text{si $x=0$,} \\
    -1 & \text{si $x<0$.}
  \end{cases}
\]

D'où finalement la jolie formule :
\begin{equation}\label{eq:result}
  \left(\prod_{k=0}^n a_k\right)
  \int_0^\infty \prod_{k=0}^n \sinc(a_k x)\,dx =
  \frac{\pi}{2}\frac{1}{2^n n!}\sum_\g \eg\,\bg^n\sign(\bg) \,.
\end{equation}

\section{Application}

On choisit maintenant :
\[
  a_0=1,\quad a_1=\frac{1}{3},\quad a_2=\frac{1}{5},
  \quad\ldots,\quad \text{et }  a_7=\frac{1}{15} \,.
\]

Remarquons que :\quad $a_1+a_2+\cdots+a_6<a_0<a_1+a_2+\cdots+a_7$.

\medskip
Pour $1\le n\le6$, on a donc $\bg>0$, pour tout $\g$.
En utilisant les relations (\ref{eq:sumegbg}) et (\ref{eq:result}),
et comme $a_0=1$, on obtient alors les premières valeurs
de la suite initiale :
\[
  \int_0^\infty\prod_{k=0}^n \sinc(a_k x)\,dx = \frac{\pi}{2} \,,\quad
  \text{pour $n=0$, $1$, $\ldots$, $6$}\,.
\]

\medskip
Pour $n=7$, on a également $\bg>0$, pour tout $\g$ sauf pour
$\g':=(-1,\ldots,-1)$. En utilisant la relation (\ref{eq:sumegbg}),
et comme $\epsilon_{\g'}=-1$, on a :
\[
  \sum_\g \eg\,\bg^7\sign(\bg) =
  \sum_\g \eg\,\bg^7 - 2\epsilon_{\g'}\,b_{\g'}^7 =
  2^7 7! \prod_{k=1}^7 a_k + 2(a_0-a_1-\ldots-a_7)^7\,.
\]

Finalement, en utilisant la relation (\ref{eq:result}), et comme $a_0=1$,
on obtient :
\[
  \int_0^\infty\prod_{k=0}^7 \sinc(a_k x)\,dx = \pi \left(\frac{1}{2}
  +\frac{(1-a_1-a_2-\cdots-a_7)^7}{2^7 7!\,a_1a_2\ldots a_7}\right)\,.
\]

Voilà qui explique la monstrueuse fraction de la dernière valeur de la
suite initiale !

\end{document}
