\documentclass[a4paper,11pt]{article}
\usepackage[french]{babel}
\usepackage[np]{numprint}
\usepackage{xspace}

\usepackage[a4paper, includefoot, textwidth=6.5in, textheight=24.956cm,
            hmarginratio=1:1, vmarginratio=1:1,
            footnotesep=1.5\baselineskip]{geometry}
\usepackage[explicit]{titlesec}
\titleformat{\section}[runin]
  {\normalfont\bfseries}
  {\thesection.}{\wordsep}{#1.}
\titlespacing{\section}{0pt}{*3}{*1.5}

\usepackage{mathtools}
\usepackage[warnings-off={mathtools-colon},
            warnings-off={mathtools-overbracket},
            math-style=french]{unicode-math}
\usepackage{fontspec}
\usepackage{microtype}

\makeatletter
\def\@maketitle{%
  \begin{center}%
    \let\footnote\thanks
    {\bfseries \@title \par}%
    \vskip .5em
    \@author
  \end{center}%
  \par \vskip .5em}

\renewenvironment{abstract}%
  {\quotation\noindent\ignorespaces}%
  {\endquotation}

\newcommand{\N}{\ensuremath{\symbb{N}}\xspace}
\newcommand{\Z}{\ensuremath{\symbb{Z}}\xspace}
\newcommand{\Q}{\ensuremath{\symbb{Q}}\xspace}
\newcommand{\ip}{\ensuremath{\symfrak{p}}\xspace}
\newcommand{\ia}{\ensuremath{\symfrak{a}}\xspace}
\newcommand{\im}{\ensuremath{\symfrak{m}}\xspace}
\AtBeginDocument{\let\ge\geqslant \let\le\leqslant}
\makeatother

\usepackage[unicode,hyperfootnotes=false,hidelinks]{hyperref}
\hypersetup{%
  pdftitle={Infinité des nombres premiers},
  pdfauthor={Lionel Vidal}
}
\usepackage{bookmark}
\title{Infinité des nombres premiers}
\author{Lionel \bsc{V\kern-1pt idal}}

\begin{document}
\maketitle
\begin{abstract}
  Où quelques démonstrations de l'infinité des nombres premiers servent
  de prétextes à des digressions mathématiques que l'on espère, c'est le
  seul enjeu, esthétiques.
\end{abstract}

\section{Euclide}

La plus ancienne démonstration connue de l'infinité des nombres premiers
est due à Euclide : s'il n'y a qu'un nombre fini $r$ de nombres premiers,
et sachant qu'il en existe au moins un, on peut construire le nombre
$n=p_1\ldots p_r+1$. Ce nombre est supérieur à $1$, donc il admet un
diviseur premier~$p$. Mais si $p$ était l'un des $p_i$, il diviserait le
produit $p_1\ldots p_r$ et $n$, donc $1$, ce qui est absurde.

Le nombre $p_1\ldots p_r+1$ n'est pas nécessairement premier (par exemple
$2\times\ldots\times 13+1=30031=59\times509$), mais est premier avec chaque
premier de l'ensemble supposé fini. De multiples constructions d'un tel
nombre sont possibles et proposent de nouvelles démonstrations se ramenant
\emph{in fine} à l'idée originelle d'Euclide.

\medskip
Par exemple,\footnote{%
  Preuve de S.~Northshield, {\slshape American Mathematical Monthly\/} (2015).}
si les nombres premiers sont en nombre fini :
\[
  0 < \prod_p \sin \left(\frac{\pi}{p}\right)
  = \prod_p \sin\left(\frac{\pi(1+2\prod_{p'}p')}{p}\right) = 0\;.
\]
L'inégalité est vraie car les facteurs du produit sont strictement
positifs ; la première égalité est vraie car le produit de tous les premiers
$p'$ est divisible par $p$ et donc l'argument du sinus est congru à $\pi/p$
modulo~$2\pi$ ; et la dernière égalité est vraie car l'entier au numérateur
est divisible par un nombre premier et donc au moins un terme du produit est
nul !

\medskip
Un formalisme élégant\footnote{
  Preuve de H.~Furstenberg, {\slshape American Mathematical Monthly\/} (1955).}
peut éviter une construction explicite :
on définit sur \Z une topologie en considérant comme ouverts l'ensemble
vide et tout ensemble d'entiers~$\symcal{O}$, tel que pour tout
$a\in\symcal{O}$, il existe une progression arithmétique $\symcal{A}$
telle que $a\in\symcal{A}\subset\symcal{O}$. En effet :
\begin{itemize}
\item l'ensemble vide et $\Z=0+1\Z$ sont ouverts ;
\item une union d'ouverts est trivialement ouverte ;
\item une intersection finie d'ouverts est ouverte car si
  $a\in\bigcap_{i=1}^n \symcal{O}_i$, il existe $n$ entiers $m_i\in\N^*$,
    tel que $a+m_i\Z\subset\symcal{O}_i$,
    et donc $a+m_1\ldots m_n\Z \subset \bigcap_{i=1}^n \symcal{O}_i$.
\end{itemize}
\smallskip
Pour cette topologie, les progressions arithmétiques sont évidemment
ouvertes mais aussi fermées :
\[
  \forall a\in\Z, \forall m\in\N^*,
  a+m\Z = \Z\;\backslash\bigcup_{i=1}^{m-1}(a+i+m\Z) \;.
\]
Si les nombres premiers sont en nombre fini, alors l'ensemble $\bigcup p\Z$
est fermé comme union finie de fermés. Mais comme tout nombre différent de
$1$ et $-1$ admet un diviseur premier, le complémentaire de cet ensemble est
$\{-1; 1\}$ qui n'est pas un ouvert.

Le point clé de cette preuve est le fait que
$\bigcap_p (\Z\,\backslash\,p\Z) = \{-1; 1\}$ ne contient pas de progression
arithmétique. Si les nombres premiers étaient en nombre fini $r$,
cet ensemble contiendrait la progression $(1+p_1\ldots p_r) \Z$, chaque $p_i$
étant premier avec $1+p_1\ldots p_r$ : on retrouve Euclide !

\section{Pierre de Fermat}

Une autre idée est de construire une suite infinie de nombres premiers entre
eux deux à deux : chacun de ces nombres admet alors au moins un facteur
premier qui n'est commun à aucun des autres, et on en déduit l'existence
d'une infinité de nombres premiers.

Par exemple, les nombres de Fermat $F_n=2^{2^n}+1$ sont premiers entre eux
deux à deux. En effet, pour $n\ge 1$ :
\[
  F_n=\prod_{d=0}^{n-1}F_d+2 \;.
\]
Mais alors, si $p$ est un diviseur premier commun de $F_d$ et $F_n$,
$p$ divise $2$, donc vaut $2$, ce qui contredit l'imparité des $F_n$.

\smallskip
La formule précédente se montre par une récurrence très simple :
$F_1=5=F_0+2$ et
\[
  \prod_{d=0}^n F_d = \left(\prod_{d=0}^{n-1}F_d\right) F_n
  = (F_n-2)F_n = (2^{2^n}-1)(2^{2^n}+1) = 2^{2^{n+1}}-1 = F_{n+1}-2 \;.
\]

Toute démonstration aurait été inutile si, comme le pensait Fermat, les
$F_n$ étaient tous premiers. C'est vrai pour $F_0=3$, $F_1=5$, $F_2=17$,
$F_3=257$ et $F_4=\np{65 537}$, mais $F_5=641\times \np{6 700 417}$.
On ne connait d'ailleurs, à ce jour, aucun nombre de Fermat premier
autre que ceux juste cités.

\section{Marin Mersenne et Joseph-Louis Lagrange}

À partir d'un nombre premier quelconque, si l'on parvient à exhiber un
autre premier strictement plus grand, on aura prouvé l'infinité de ces
nombres. Soit donc $p$ un nombre premier et $q$ un facteur premier du nombre
de Mersenne $2^p-1$. On a donc $2^p\equiv 1\pod{q}$, et comme $p$ est
premier, $2$ est d'ordre $p$ dans le sous-groupe multiplicatif $(\Z/q\Z)^*$
du corps $\Z/q\Z$. Mais d'après le théorème de Lagrange, l'ordre de
l'élément d'un groupe divise l'ordre de ce groupe, soit $p\mid (q-1)$. Et
donc $p<q$.

\section{Leonhard Euler}

En 1737, Euler montre que la série des inverses des nombres premiers,
\[
  \frac{1}{2}+\frac{1}{3}+\frac{1}{5}+\frac{1}{7}+\frac{1}{11}+\cdots
\]
est divergente, ce qui montre bien plus que l'infinité des nombres premiers :
cela donne une idée de leur \emph{densité} dans l'ensemble des entiers.
Ils sont plus denses par exemple que les carrés, au sens où la série des
inverses des carrés converge.

Pour tout entier $n\ge 2$, notons :
\[
  P_n \coloneq \prod_{p\le n}\frac{1}{1-\frac{1}{p}}
  = \prod_{p\le n}\left(1+\frac{1}{p}+\frac{1}{p^2}+\cdots\right) \;.
\]

Comme tout entier naturel non nul se décompose de manière unique, à
l'ordre près, en produit de facteurs premiers, les termes du développement
de $P_n$ comprennent l'inverse de chaque entier $k\le n$ ; en notant $q$ le
plus grand premier inférieur ou égal à $n$, il vient :
\begin{align*}
 P_n &= \left(1+\frac{1}{2}+\frac{1}{2^2}+\cdots\right)
        \left(1+\frac{1}{3}+\frac{1}{3^2}+\cdots\right)\ldots
        \left(1+\frac{1}{q}+\frac{1}{q^2}+\cdots\right) \\
     &= 1+\frac{1}{2}+\frac{1}{3}+\frac{1}{2\times2}+\frac{1}{5}+
        \frac{1}{2\times3}+\frac{1}{7}+\cdots
      > \sum_{k=1}^n \frac{1}{k} > \int_1^n \frac{dx}{x} = \log n \;.
\end{align*}
D'où $\displaystyle\lim_{n\to+\infty}P_n=+\infty$,
et donc déjà l'infinité des nombres premiers ! De plus :
\[
  \log(\log n) < \log P_n
  = \sum_{p\le n} \log\left(1+\frac{1}{p}+\frac{1}{p^2}+\cdots\right)
  \le \sum_{p\le n} \left(\frac{1}{p}+\frac{1}{p^2}+\cdots\right)
  = \sum_{p\le n} \frac{1}{p} + R_n \;,
\]
où $R_n$ est défini par :
\begin{align*}
  R_n &\coloneq \sum_{p\le n} \left(\frac{1}{p^2}+\cdots\right)
      = \sum_{p\le n} \frac{1}{p^2}\left(1+\frac{1}{p}+\cdots\right)
      = \sum_{p\le n} \frac{1}{p^2}\frac{1}{1-\frac{1}{p}}
      = \sum_{p\le n} \frac{1}{p(p-1)} \\
      &\le \sum_{k=2}^n \frac{1}{k(k-1)}
      = \sum_{k=2}^n \left(\frac{1}{k-1}-\frac{1}{k}\right)
      = 1-\frac{1}{n} < 1 \;.
\end{align*}
D'où finalement :
$\displaystyle\log(\log n) - 1 < \sum_{p\le n} \frac{1}{p}\;$,
ce qui montre que la série $\displaystyle\sum\frac{1}{p}$ diverge.

\section{Euler encore, revu par Paul Erd\H{o}s}

Donnons une autre belle preuve, inventée par Erd\H{o}s, de la divergence
de la série des inverses des nombres premiers.

Soient $p_1$, $p_2$, $\ldots$, les nombres premiers en ordre
croissant. Supposons que la série de leurs inverses soit convergente.
Il existe alors un entier naturel $k$ tel que :
\[
  \sum_{i\ge k+1} \frac{1}{p_i} < \frac{1}{2} \;.
\]
On appellera $p_1$, $\ldots$, $p_k$, les \emph{petits} nombres premiers, et
$p_{k+1}$, $p_{k+2}$, $\ldots$, les \emph{grands} nombres premiers. Soit $N$
un nombre entier naturel non nul. On note $G$ le nombre d'entiers naturels
$n\le N$ divisibles par au moins un grand nombre premier et $P$ le nombre
d'entiers naturels $n\le N$ qui n'ont que des petits diviseurs premiers
(ainsi $0$ est compté dans $G$ et $1$ n'est pas compté).

Majorons alors $G$ et $P$.

D'une part, $\lfloor N/p_i\rfloor$ dénombre les entiers $n\le N$
qui sont des multiples de $p_i$. D'où :
\[
  G \le \sum_{i\ge k+1} \left\lfloor\frac{N}{p_i}\right\rfloor
    \le \sum_{i\ge k+1} \frac{N}{p_i} < \frac{N}{2} \;.
\]

D'autre part, tout entier $n\le N$ qui n'admet que des petits
diviseurs premiers peut s'écrire $n=a_n b_n^2$, où $a_n$ est sans facteur
carré et donc produit, éventuellement vide, de petits nombres premiers
deux à deux distincts. Il y a donc $2^k$ facteurs possibles pour $a_n$,
et comme $b_n\le\sqrt{N}$ :
\[
  P \le 2^k \sqrt{N} \;.
\]

D'où : $\displaystyle G+P=N < \frac{N}{2} + 2^k \sqrt{N}$, soit
$\sqrt{N}<2^{k+1}$, ce qui contredit le choix arbitraire de~$N$.

\section{Euler toujours, et la fonction de Zéta de Bernhard Riemann}

\[
  \prod_p \frac{1}{1-\frac{1}{p^2}} = \zeta(2) = \frac{\pi^2}{6}
\]

Le membre de droite est irrationnel, donc le produit de gauche a un
nombre infini de termes !

\smallskip
La fonction Zéta de Riemann est définie sur $]1, +\infty[$ par
$\displaystyle\zeta(x) \coloneq \sum_{n=1}^{+\infty} \frac{1}{n^x}$.
Comme pour $s>1$,
\[
  \frac{1}{1-\frac{1}{p^s}} = 1+\frac{1}{p^s}+\frac{1}{p^{2s}}+
                              \frac{1}{p^{3s}}+\cdots \;,
\]
en considérant tous les nombres premiers $p$ inférieurs à un nombre premier
$q$ fixé :
\[
  \left\vert\zeta(s)-\prod_{p\le q} \frac{1}{1-\frac{1}{p^s}}\right\vert
  < \sum_{n=q+1}^{+\infty} \frac{1}{n^s} \;.
\]
En effet, comme chaque entier se décompose de façon unique en produit de
facteurs premiers, le produit partiel, une fois développé en utilisant la
relation précédente, ne comprend que des termes de type $n^{-s}$, où $n$
est un produit de premiers au plus égaux à $q$. Donc seuls les entiers
plus grands que $q$ peuvent ne pas apparaître dans le produit partiel,
ce qui justifie l'inégalité.
La valeur absolue précédente tend vers $0$ quand $q$ tend vers l'infini,
et donc pour tout $s>1$ :
\[
  \prod_p \frac{1}{1-\frac{1}{p^s}} = \zeta(s) \;.
\]

Déterminons maintenant la valeur de $\zeta(2)$. Pour une preuve courte,
sortons l'artillerie lourde : dans l'espace de Hilbert $L^2[0,1]$ muni de
la base orthonormée $(e_n(x)=\exp(2i\pi n x))_{n\in\Z}$, on applique à la
fonction identité la formule de Parseval :
\[
  \langle f,f\rangle=\sum_{n\in\Z}\vert\langle f,e_n\rangle\vert^2 \;
  \mbox{, où }\; \langle f,g\rangle \coloneq \int_0^1f\bar g \;.
\]

On calcule très facilement :
\begin{gather*}
  \langle f,f \rangle = \int_0^1 x^2\,dx = \frac{1}{3} \;,\quad
  \langle f,e_0 \rangle = \int_0^1 x\,dx = \frac{1}{2} \;, \\
  \mbox{et pour $n$ non nul, }
  \langle f,e_n \rangle = \int_0^1 x e^{-2i\pi n x}\,dx
  =\left[x\frac{e^{-2i\pi n x}}{-2i\pi n}\right]_0^1
  -\underbrace{\int_0^1 \frac{e^{-2i\pi n x}}{-2i\pi n}\,dx}_{=0}
  =-\frac{1}{2i\pi n} \,.
\end{gather*}
D'où\quad
$\displaystyle\frac{1}{3}=\frac{1}{4}+\sum_{n\in\Z^*}\frac{1}{4\pi^2 n^2}$,
\quad et donc :\quad
$\displaystyle\zeta(2)=\sum_{n=1}^{+\infty} \frac{1}{n^2}=\frac{\pi^2}{6}$.

\medskip
Pour être complet, il faudrait encore démontrer que $\pi^2$ est
irrationnel\dots{} ce qui sera admis ici !

\section{Dénombrement}

Peut-être la démonstration la plus naturelle : on compte le nombre de
décompositions possibles en produit de facteurs premiers plus petits qu'un
entier donné. Si on note $\pi(n)$ le nombre de nombres premiers inférieurs
ou égaux à $n$, et $p_1$, $p_2$, $\ldots$, les nombres premiers en ordre
croissant, tout entier $k$, $0<k\le n$, s'écrit :
\[
  k = \prod_{i=1}^{\pi(n)} p_i^{e_i}\;,
  \mbox{\enspace où pour tout $i$, $e_i$ est un entier naturel.}
\]
D'où : $\displaystyle\sum_{i=1}^{\pi(n)} e_i\log p_i \le\log n$,
et grossièrement,
$\displaystyle  e_i\le\frac{\log n}{\log p_i} \le \log_2 n$.
En dénombrant alors toutes les décompositions, chaque $p_i$ offre
au plus $\log_2 n$ choix, et donc
$\displaystyle n \le (\log_2 n + 1)^{\pi(n)}$. On en déduit :
\[
  \pi(n) \ge \frac{\log n}{\log(\log_2 n +1)}\;,\quad\mbox{et donc :}\quad
  \lim_{n\to+\infty}\pi(n) = +\infty\;.
\]

\section{Richard Dedekind}

Pour finir, montons de quelques barreaux sur l'échelle de Jacob\dots

\noindent
Beaucoup d'anneaux d'entiers de corps de nombres ne sont pas factoriels,
et leurs éléments n'admettent donc pas une unique décomposition en produit de
facteurs premiers. Pour pallier cet inconvénient, dans le cadre de travaux
sur l'équation de Fermat, Dedekind développe la notion d'anneau,
dit de Dedekind, plus général qu'un anneau principal :
\begin{quotation}
  \noindent\itshape
  un anneau de Dedekind est un anneau Nœthérien, intégralement clos,
  dans lequel tout idéal premier non nul est maximal.
\end{quotation}

\noindent
On montre que dans un tel anneau, tous les \emph{idéaux} admettent une
unique décomposition en \emph{idéaux premiers}, et que tout anneau
d'entiers de corps de nombres est un anneau de Dedekind.

\medskip
Considérons alors un anneau de Dedeking $R$ qui n'admette qu'un nombre fini
d'idéaux premiers non nuls, $\ip_1$, $\dots$, $\ip_n$, et montrons que dans
ce cas, $R$ est principal.

En notant $R_{\ip_1}$ le localisé de $R$ en $\ip_1$, comme le module
$\ip_1R_{\ip_1}$ est non nul, d'après le lemme de Nakayama,
$\ip_1^2 R_{\ip_1} \subsetneq \ip_1 R_{\ip_1}$,
et donc $\ip_1^2 \subsetneq \ip_1$.
Soit alors $y_1\in\ip_1\backslash\,\ip_1^2$.

Remarquons que si $\ia$ et $\im$ sont des idéaux de $R$, tels que
$\ia\not\subset\im$ et que $\im$ soit maximal, alors $\ia+\im=R$ et
les idéaux $\ia$ et $\im$ sont premiers entre eux.
Comme $\ip_2$, $\dots$, $\ip_n$ sont maximaux, les idéaux
$\ip_1^2$, $\ip_2$, $\dots$, $\ip_n$ sont donc premiers entre eux
deux à deux.
D'après le lemme des restes chinois, il existe alors $x_1\in R$, tel que
$x_1\equiv y_1 \pod{\ip_1^2}$, et $x_1\equiv 1 \pod{\ip_k}$,
pour $2 \le k \le n$.

Comme $R$ est de Dedekind, $\langle x_1\rangle$ se décompose en
$\langle x_1\rangle=\ip_1^{e_1}\dots\ip_n^{e_n}$, où $e_1$, $\dots$, $e_n$
sont des entiers naturels.
Mais pour $2\le k\le n$, $x_1\not\in\ip_k$, donc $e_k=0$.
De plus, comme $x_1\in\ip_1\backslash\,\ip_1^2$, on a $e_1=1$.
D'où $\ip_1=\langle x_1\rangle$ et $\ip_1$ est donc principal.
On montre de même que $\ip_2$, $\dots$, $\ip_n$ sont principaux.

Mais tout idéal $\ia$ de $R$ se décompose en
$\ia=\ip_1^{a_1}\dots\ip_n^{a_n}=
 \langle x_1\rangle^{a_1}\dots\langle x_n\rangle^{a_n}=
 \langle x_1^{a_1}\dots x_n^{a_n}\rangle$,
et l'anneau $R$ est donc principal.

\medskip
Considérons alors l'anneau $\Z[\sqrt{-5}]$, anneau de Dedekind des entiers
de $\Q[\sqrt{-5}]$. L'élément $1+\sqrt{-5}$ est irréductible, car la norme
d'un de ses diviseurs non trivial devrait être un diviseur non trivial de sa
norme 6, et les équations $a^2+5b^2=2$ et $a^2+5b^2=3$ n'ont pas de solution
dans~$\Z$.

Mais comme $(1+\sqrt{-5})(1-\sqrt{-5})=2\times3$, l'anneau $\Z[\sqrt{-5}]$
n'est pas principal, car sinon l'élément $1+\sqrt{-5}$ irréductible
diviserait $2$ ou $3$, et en prenant les normes, $6$ diviserait $4$ ou $9$.

Donc d'après ce qui précède, les nombres premiers sont en nombre infini !

\medskip
C'est ma preuve préférée : moralement, il est nécessaire que les nombres
premiers soient en nombre infini pour que la théorie des nombres soit
aussi intéressante et complexe !

\end{document}
